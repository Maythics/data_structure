\documentclass[UTF8]{ctexart}
\usepackage{geometry, CJKutf8}
\geometry{margin=1.5cm, vmargin={0pt,1cm}}
\setlength{\topmargin}{-1cm}
\setlength{\paperheight}{29.7cm}
\setlength{\textheight}{25.3cm}
\setlength{\headheight}{13pt}

% useful packages.
\usepackage{amsfonts}
\usepackage{amsmath}
\usepackage{amssymb}
\usepackage{amsthm}
\usepackage{enumerate}
\usepackage{graphicx}
\usepackage{multicol}
\usepackage{fancyhdr}
\usepackage{layout}
\usepackage{listings}
%\usepackage{courier}
\usepackage{float, caption}

\lstset{
    basicstyle=\ttfamily, basewidth=0.5em
}

% some common command
\newcommand{\dif}{\mathrm{d}}
\newcommand{\avg}[1]{\left\langle #1 \right\rangle}
\newcommand{\difFrac}[2]{\frac{\dif #1}{\dif #2}}
\newcommand{\pdfFrac}[2]{\frac{\partial #1}{\partial #2}}
\newcommand{\OFL}{\mathrm{OFL}}
\newcommand{\UFL}{\mathrm{UFL}}
\newcommand{\fl}{\mathrm{fl}}
\newcommand{\op}{\odot}
\newcommand{\Eabs}{E_{\mathrm{abs}}}
\newcommand{\Erel}{E_{\mathrm{rel}}}

\begin{document}

\pagestyle{fancy}
\fancyhead{}
\lhead{章翰宇 3220104133}
\chead{数据结构与算法第六次作业}
\rhead{Nov.10th, 2024}

\section{remove函数实现的阐述}

\begin{enumerate}
  \item 首先,若对象是空树,则直接return。其次,若对象比当前节点小,则继续向左找,比当前大,向右找。找到对象后,如果该待删除对象只有一侧子树甚至没有子树,都很容易实现,按照原版代码即可(直接把子树接管权交给父辈,然后删除该节点即可)。我们只关心当待删除对象的左右子树均非空时的情况。

  \item 此时,需要先把它的继任者找出来,这个找继任者的操作就在detachMin里完成。detachMin负责传回将继任者找到并且无痕地取出来,由于继任者要求是右子树的最小者,一路往左就行。找到继任者后,要将继任者从树上剥离下来,但继任者可能也有子辈,于是将其子辈接回断枝,即:继任者子辈和继任者父辈相接。此外,别忘了先找个指针盯着这个继任者(否则就弄丢了),该指针值也作为参数传出去,被remove函数中的变量接收。\

  \item 现在remove函数拿到了剥离下来的继任者,只需要做善后工作:把删除对象的左枝右枝告诉继任者,把指向删除对象的指针改成指向继任者,然后把删除对象delete即可。\

  \item 这些结束后,在\textbf{序关系}层面已经全对了,但是在树层面还不一定平衡,下面讨论如何使树平衡。\

  \item 分析:由于仅删除了一个节点,而删除后,将会由其右子树的最小元素来替补它。显然,修改后的树的节点里,height参数需要变更的那些节点,必仅存在于被删除节点以下。CASE1:继任者没有子节点,此时,仅需继任者改变height参数;CASE2:继任者有子节点,根据继任者的定义,仅有右子树,而右子树是作为一个整体接到继任者本来的父辈身上的,因此,右子树中的所有节点的height参数无需改变,只用考察继任者本身开始往上,那些height参数需改变,因此,在detachMin递归的途中,调用balance即可。\

  \item 在remove函数处调用balance函数,传入的参数是当前节点的指针引用,目的是把remove后的该层进行balance,因为该层以下的balance问题,已经在detachMin函数中解决了。\

\end{enumerate}

\section{代码测试}

测试环境:12th Gen Intel(R) Core(TM) i5-12500H   2.50 GHz,在WSL中进行测试

首先需要\textbf{接除限制}:ulimit -s unlimited

可以直接执行make run:(即沿袭上次作业,执行g++ test.cpp -o test -O2以及time ./test)

也可以分开执行,然后time ./test的输出如下(省略前面打印树的很长的输出):
\begin{verbatim}
  ------------------------------
  Empty tree

  real    0m0.942s
  user    0m0.932s
  sys     0m0.010s
\end{verbatim}

我用valgrind --leak-check=full ./test进行测试,发现没有发生内存泄露。

\end{document}

