\documentclass[UTF8]{ctexart}
\usepackage{geometry, CJKutf8}
\geometry{margin=1.5cm, vmargin={0pt,1cm}}
\setlength{\topmargin}{-1cm}
\setlength{\paperheight}{29.7cm}
\setlength{\textheight}{25.3cm}
\setlength{\headheight}{13pt}

% useful packages.
\usepackage{amsfonts}
\usepackage{amsmath}
\usepackage{amssymb}
\usepackage{amsthm}
\usepackage{enumerate}
\usepackage{graphicx}
\usepackage{multicol}
\usepackage{fancyhdr}
\usepackage{layout}
\usepackage{listings}
%\usepackage{courier}
\usepackage{float, caption}

\lstset{
    basicstyle=\ttfamily, basewidth=0.5em
}

% some common command
\newcommand{\dif}{\mathrm{d}}
\newcommand{\avg}[1]{\left\langle #1 \right\rangle}
\newcommand{\difFrac}[2]{\frac{\dif #1}{\dif #2}}
\newcommand{\pdfFrac}[2]{\frac{\partial #1}{\partial #2}}
\newcommand{\OFL}{\mathrm{OFL}}
\newcommand{\UFL}{\mathrm{UFL}}
\newcommand{\fl}{\mathrm{fl}}
\newcommand{\op}{\odot}
\newcommand{\Eabs}{E_{\mathrm{abs}}}
\newcommand{\Erel}{E_{\mathrm{rel}}}

\begin{document}

\pagestyle{fancy}
\fancyhead{}
\lhead{章翰宇 3220104133}
\chead{数据结构与算法第五次作业}
\rhead{Nov.03th, 2024}

\section{修改后remove函数实现的阐述}

首先,若对象是空树,则直接return
其次,若对象比当前节点小,则继续向左找,比当前大,向右找。找到对象后,如果该待删除对象只有一侧子树甚至没有子树,都很容易实现,按照原版代码即可。我们只关心当待删除对象的左右子树均非空时的情况。

此时,需要先把它的继任者找出来,这个找继任者的操作就在detachMin里完成。

detachMin负责传回将继任者找到并且无痕地取出来,由于继任者要求是右子树的最小者,一路往左就行。找到继任者后,要将继任者从树上剥离下来,但继任者可能也有子辈,于是将其子辈接回断的那根枝,即:继任者子辈和继任者父辈相接,别忘了先要弄个指针盯着这个继任者(否则就弄丢了),该指针值也作为参数传出去,被remove函数中的变量接收。

现在remove函数拿到了剥离下来的继任者,只需要做善后工作:把删除对象的左枝右枝告诉继任者,把指向删除对象的指针改成指向继任者,然后把删除对象delete即可。


\section{测试程序的设计思路}

先测试正常的情况(即用户的使用不会触发抛出异常警告),为了验证remove正确操作,需要验证这些情况:\
1 删除对象不存在\
2 删除对象恰好是叶节点\
3 删除对象的继任者恰好是叶节点\
4 删除对象的继任者还有后继\
5 对一个空树进行删除\

\begin{enumerate}

    \item 我首先insert了一些更加多的元素,使得树变得更加复杂,最终,形成的BST树如下:
    \begin{verbatim}
                     10
                     |
             5-----------------15
             |                 |
        3------------7   11------------18
                         |
                          --13
                            |
                          12---14
    \end{verbatim}
    \item 接着,进行remove操作,首先测试普通的remove(5),对应的是:“接替删除对象的那个节点”恰好是叶子的情况,理想结果是成为树:
    \begin{verbatim}
                     10
                     |
             7-----------------15
             |                 |
        3-----           11------------18
                         |
                          --13
                            |
                          12---14
    \end{verbatim}

    \item 接着,测试remove一个不存在的数据6,希望的结果是什么都不发生。

    \item 然后,测试remove根节点10,预期的结果是,由右子树的最小者(11)来接替成为根节点,然后11的子辈被11的父辈接管,即:
    \begin{verbatim}
                     11
                     |
             7-----------------15
             |                 |
        3-----           13------------18
                         |
                      12---14
    \end{verbatim}
    \item 再测试被删除对象恰好就是叶节点的情况,remove(18):
    \begin{verbatim}
                     11
                     |
             7-----------------15
             |                 |
        3-----           13-----
                         |
                      12---14
    \end{verbatim}
    \item 最后测试对一棵空树进行remove,是否会出问题。理想情况是什么都不做,打印该树,仍然输出“empty tree”



\end{enumerate}

\section{测试的结果}

在工作空间下使用命令\texttt{make run},输出将会打印到terminal中,如下

\begin{verbatim}
Initial Tree:
3
5
7
10
11
12
13
14
15
18
Minimum element: 3
Maximum element: 18
Contains 7? Yes
Contains 20? No
Tree after removing 5:
3
7
10
11
12
13
14
15
18
Tree after removing 6:
3
7
10
11
12
13
14
15
18
Tree after removing 10:
3
7
11
12
13
14
15
18
Tree after removing 18:
3
7
11
12
13
14
15
Treeddddd after making empty:
Empty tree
Is tree empty? Yes
Empty Tree after removing 7:
Empty tree
Copied Tree (bst3):
1
2
3
Assigned Tree (bst4):
1
2
3
Moved Tree (bst5):
1
2
3
Move Assigned Tree (bst6):
1
2
3
\end{verbatim}

我用valgrind --leak-check=full ./test\_BST进行测试,发现没有发生内存泄露:\
ERROR SUMMARY: 0 errors from 0 contexts (suppressed: 0 from 0)

以上输出,请参见测试remove程序的思路一节,我已经把生成的期望的树阐述了,由于测试代码中是“左中右”顺序打印的树,可以验证,和我设计思路中期望的相同,达到了效果。

\section{异常测试报告}

将最后一行代码解注释,即测试对一棵空树进行bst7.findMax();测试,将会触发throw UnderflowException\{ \},输出的确会如此,产生1 error。

\end{document}

