\documentclass[UTF8]{ctexart}
\usepackage{geometry, CJKutf8}
\geometry{margin=1.5cm, vmargin={0pt,1cm}}
\setlength{\topmargin}{-1cm}
\setlength{\paperheight}{29.7cm}
\setlength{\textheight}{25.3cm}
\setlength{\headheight}{13pt}

% useful packages.
\usepackage{amsfonts}
\usepackage{amsmath}
\usepackage{amssymb}
\usepackage{amsthm}
\usepackage{enumerate}
\usepackage{graphicx}
\usepackage{multicol}
\usepackage{fancyhdr}
\usepackage{layout}
\usepackage{listings}
%\usepackage{courier}
\usepackage{float, caption}

\lstset{
    basicstyle=\ttfamily, basewidth=0.5em
}

% some common command
\newcommand{\dif}{\mathrm{d}}
\newcommand{\avg}[1]{\left\langle #1 \right\rangle}
\newcommand{\difFrac}[2]{\frac{\dif #1}{\dif #2}}
\newcommand{\pdfFrac}[2]{\frac{\partial #1}{\partial #2}}
\newcommand{\OFL}{\mathrm{OFL}}
\newcommand{\UFL}{\mathrm{UFL}}
\newcommand{\fl}{\mathrm{fl}}
\newcommand{\op}{\odot}
\newcommand{\Eabs}{E_{\mathrm{abs}}}
\newcommand{\Erel}{E_{\mathrm{rel}}}

\begin{document}

\pagestyle{fancy}
\fancyhead{}
\lhead{章翰宇 3220104133}
\chead{数据结构与算法项目作业}
\rhead{Dec.08st, 2024}

\section{布谷鸟哈希 程序设计}

参考了课本的代码,并且在其基础上进行了简化。首先,概述这里的布谷鸟哈希insert实现流程(与上课讲的双表型有所不同):

\begin{enumerate}

  \item 初始化一个哈希表,长度可以设置(默认大小101),其中所有位置都是空的。备用哈希函数的个数可以设置(默认3种)
  \item 当用insert方法插入元素时,会根据元素的类型去调用对应的哈希函数(我这里有整型和字符串两种),根据计算得到的位置尝试存入
  \item 如果该位置已经有元素占用了,使用下一个备选哈希函数……若存在一个备用哈希函数,使得该元素经其映射得到的位置是空的,则将该哈希函数填入之,返回。否则,转入下一步
  \item 踢除操作:任选一个备选哈希函数,该元素强行进入经其映射到的那个位置里,并把原来的老元素踢出来,再对被踢出来的老元素使用insert方法,如此循环。
  \item 如果发生踢除操作的次数大于阈值,则认为表太小,调用rehash方法扩增,更换新的哈希函数,再把表的大小翻为原大小的两倍以上。然后,把原来的元素(包括刚才填不进去的那个元素),全部insert一遍到新表里

\end{enumerate}

代码说明:

1. 一共就只有一张表(这样的实现一般来说会比双表更省空间)

2. 与课本代码不同,我简化了rehash的判别,如果在某一次insert之中,造成的碰撞次数超过阈值(阈值我设为当前元素总数的四分之一,和100取min的值)就rehash(无需像课本代码有个内置的变量去记录累计量)

3. 与课本代码不同,调整了哈希函数的生成规则

4. 增加了打印哈希函数、打印哈希表的public函数,便于可视化结果与debug


\section{测试思路}

基本功能正确与否:测试两种类型(字符串和整型)分别作为基本元素时,哈希表是否工作正常,首先是两个含有可视化的函数,会显示出此次的哈希函数是哪些,也会在每一步都打印出此时的表的大小、此时存有的元素的个数,也会打出整张表。此外,也会在触发布谷鸟踢除时,显示出“kick”了哪个元素,在触发rehash时会打印“rehash”,方便检查。也测试了remove函数以及contain函数的正确实现。

效率考察:

在测试时使用chrono进行计时,执行前后的时间作差,使用print\_result函数打印出时间,包括check的结果即可。

注意,这里的计时统一用\textbf{建堆后开始到排序结束}的时间。

\section{测试效果}

可直接使用make run 命令编译且运行。

输出形如:
\begin{verbatim}
  **Random Input**
  sort_heap: 0.157933
  my_sort: 0.15796
  check result: 1
 **Increasing Input**
  sort_heap: 0.063219
  my_sort: 0.0688413
  check result: 1
 **Decreasing Input**
  sort_heap: 0.0810255
  my_sort: 0.0808056
  check result: 1
 **Random with Repeat Input**
  sort_heap: 0.152754
  my_sort: 0.157996
  check result: 1
\end{verbatim}

在未加入-O2优化时,如下(单位:second):

\begin{table}[H]
  \begin{center}
  \caption{运行时间对比}
  \begin{tabular}{l|l|l}
  \label{table1}
  \textbf{输入序列} & \textbf{my\_sort time} & \textbf{std::sort\_heap time}\\
  \hline
  随机 & 0.733434 & 1.02845\\
  递增 & 0.53948 & 0.851013\\
  递减 & 0.542546 & 0.865929\\
  随机(含重复) & 0.716736 & 1.0294
  \end{tabular}
  \end{center}
\end{table}

在编译时加入-O2进行优化后,如下(单位:second):

\begin{table}[H]
  \begin{center}
  \caption{运行时间对比(-O2)}
  \begin{tabular}{l|l|l}
  \textbf{输入序列} & \textbf{my\_sort time} & \textbf{std::sort\_heap time}\\
  \hline
  随机 & 0.1622 & 0.161694 \\
  递增 & 0.076275 & 0.0677776 \\
  递减 & 0.0800664 & 0.0843663 \\
  随机(含重复) & 0.160439 & 0.156313
  \end{tabular}
  \end{center}
\end{table}

用valgrind --leak-check=full ./test进行测试,发现没有发生内存泄露 \

\section{时间复杂度分析}

由于每一次弹出最大元是$\Theta(1)$的,而耗时的主要是“向下过滤”的过程,由于最糟糕的情况是向下过滤到底,即树高$h$那么多。而共操作$N$次,因此为$\Theta(N \log N)$

实际上这和标准库的方法一样,因此时间复杂度一致。但,由于我写的my\_sort函数要求输入必须\textbf{已经是个堆了,不会进行检查},而std::sort\_heap函数则会首先进行检查,如果不是堆会先建堆。这导致在“检查阶段”耗时,所以可见在表格\ref{table1}无O2优化时,我的函数在各种情况下都更快0.3秒左右,有理由猜测是检查的耗时。

\end{document}
