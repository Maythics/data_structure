\documentclass[UTF8]{ctexart}
\usepackage{geometry, CJKutf8}
\geometry{margin=1.5cm, vmargin={0pt,1cm}}
\setlength{\topmargin}{-1cm}
\setlength{\paperheight}{29.7cm}
\setlength{\textheight}{25.3cm}
\setlength{\headheight}{13pt}

% useful packages.
\usepackage{amsfonts}
\usepackage{amsmath}
\usepackage{amssymb}
\usepackage{amsthm}
\usepackage{enumerate}
\usepackage{graphicx}
\usepackage{multicol}
\usepackage{fancyhdr}
\usepackage{layout}
\usepackage{listings}
%\usepackage{courier}
\usepackage{float, caption}

\lstset{
    basicstyle=\ttfamily, basewidth=0.5em
}

% some common command
\newcommand{\dif}{\mathrm{d}}
\newcommand{\avg}[1]{\left\langle #1 \right\rangle}
\newcommand{\difFrac}[2]{\frac{\dif #1}{\dif #2}}
\newcommand{\pdfFrac}[2]{\frac{\partial #1}{\partial #2}}
\newcommand{\OFL}{\mathrm{OFL}}
\newcommand{\UFL}{\mathrm{UFL}}
\newcommand{\fl}{\mathrm{fl}}
\newcommand{\op}{\odot}
\newcommand{\Eabs}{E_{\mathrm{abs}}}
\newcommand{\Erel}{E_{\mathrm{rel}}}

\begin{document}

\pagestyle{fancy}
\fancyhead{}
\lhead{章翰宇 3220104133}
\chead{数据结构与算法第12周作业}
\rhead{Dec.01st, 2024}

\section{程序设计思路}



对于一个最大堆(最大的元素在根部的)而言,可以通过依次弹射最大元的方法,实现排序,因此总体思路为,每一次都将最大元弹射出去排序,然后将剩下的数据重新调整成堆即可。

下面讨论细节:弹射操作非常容易,因为对于一个最大堆而言,其根节点就是最大元,至于如何将剩下的数据重新变成堆需要着重处理。可以先将数组末尾的元素安置到堆的根部去,但是这并不是堆(因为不符合堆序性),所以通过“下滤”操作,如果这个元素比其孩子还小,那么就与孩子进行交换,就好像“下沉”一样,如此不断与孩子比较,直到下不去为止。

流程:
\begin{enumerate}
  \item 弹出最大元,最末位补充,即两者进行swap。(因为最大的元素,进行递增排序后恰会排在末尾)

  \item 调用Downward\_sift函数把剩余的数据变成一个正确的堆,由于排除根节点外的元素,都已符合堆序性,因此仅需跟踪该根节点的下沉过程。

  \item 再次重复前两步操作。

\end{enumerate}


\section{测试思路}

其中check函数只需要对比使用标准库的方法和使用我自己的方法,生成的两个vector是否一致即可,可传入引用以避免复制。

\begin{enumerate}

  \item 长度1500000的随机序列
  \item 长度1500000的增序序列
  \item 长度1500000的降序序列
  \item 长度1500000的部分元素重复序列(只需要限制随机数产生的范围,自动就能得到含重复的元素的随机序列)

\end{enumerate}

在测试时使用chrono进行计时,执行前后的时间作差,使用print\_result函数打印出时间,包括check的结果即可。

\section{测试效果}

可直接使用make run 命令编译且运行。

输出形如:
\begin{verbatim}
  **Random Input**
  sort_heap: 0.157933
  my_sort: 0.15796
  check result: 1
 **Increasing Input**
  sort_heap: 0.063219
  my_sort: 0.0688413
  check result: 1
 **Decreasing Input**
  sort_heap: 0.0810255
  my_sort: 0.0808056
  check result: 1
 **Random with Repeat Input**
  sort_heap: 0.152754
  my_sort: 0.157996
  check result: 1
\end{verbatim}

在未加入-O2优化时,如下(单位:second):

\begin{table}[H]
  \begin{center}
  \caption{运行时间对比}
  \begin{tabular}{l|l|l}
  \label{table1}
  \textbf{输入序列} & \textbf{my\_sort time} & \textbf{std::sort\_heap time}\\
  \hline
  随机 & 0.733434 & 1.02845\\
  递增 & 0.53948 & 0.851013\\
  递减 & 0.542546 & 0.865929\\
  随机(含重复) & 0.716736 & 1.0294
  \end{tabular}
  \end{center}
\end{table}

在编译时加入-O2进行优化后,如下(单位:second):

\begin{table}[H]
  \begin{center}
  \caption{运行时间对比(-O2)}
  \begin{tabular}{l|l|l}
  \textbf{输入序列} & \textbf{my\_sort time} & \textbf{std::sort\_heap time}\\
  \hline
  随机 & 0.1622 & 0.161694 \\
  递增 & 0.076275 & 0.0677776 \\
  递减 & 0.0800664 & 0.0843663 \\
  随机(含重复) & 0.160439 & 0.156313
  \end{tabular}
  \end{center}
\end{table}

用valgrind --leak-check=full ./test进行测试,发现没有发生内存泄露 \

\section{时间复杂度分析}

由于每一次弹出最大元是$\Theta(1)$的,而耗时的主要是“向下过滤”的过程,由于最糟糕的情况是向下过滤到底,即树高$h$那么多。而共操作$N$次,因此为$\Theta(N \log N)$

实际上这和标准库的方法一样,因此时间复杂度一致。但,由于我写的my\_sort函数要求输入必须\textbf{已经是个堆了,不会进行检查},而std::sort\_heap函数则会首先进行检查,如果不是堆会先建堆。这导致在“检查阶段”耗时,所以可见在表格\ref{table1}无O2优化时,我的函数在各种情况下都更快0.3秒左右,有理由猜测是检查的耗时。

\end{document}
