\documentclass[UTF8]{ctexart}
\usepackage{geometry, CJKutf8}
\geometry{margin=1.5cm, vmargin={0pt,1cm}}
\setlength{\topmargin}{-1cm}
\setlength{\paperheight}{29.7cm}
\setlength{\textheight}{25.3cm}

% useful packages.
\usepackage{amsfonts}
\usepackage{amsmath}
\usepackage{amssymb}
\usepackage{amsthm}
\usepackage{enumerate}
\usepackage{graphicx}
\usepackage{multicol}
\usepackage{fancyhdr}
\usepackage{layout}
\usepackage{listings}
\usepackage{float, caption}

\lstset{
    basicstyle=\ttfamily, basewidth=0.5em
}

% some common command
\newcommand{\dif}{\mathrm{d}}
\newcommand{\avg}[1]{\left\langle #1 \right\rangle}
\newcommand{\difFrac}[2]{\frac{\dif #1}{\dif #2}}
\newcommand{\pdfFrac}[2]{\frac{\partial #1}{\partial #2}}
\newcommand{\OFL}{\mathrm{OFL}}
\newcommand{\UFL}{\mathrm{UFL}}
\newcommand{\fl}{\mathrm{fl}}
\newcommand{\op}{\odot}
\newcommand{\Eabs}{E_{\mathrm{abs}}}
\newcommand{\Erel}{E_{\mathrm{rel}}}

\begin{document}

\pagestyle{fancy}
\fancyhead{}
\lhead{章翰宇 3220104133}
\chead{数据结构与算法第四次作业}
\rhead{Oct.18th, 2024}

\section{测试程序的设计思路}



\begin{enumerate}

    \item 我首先创建了一个链表list1,测试一下默认的构造函数,即理应得到仅有head与tail,size为0的一个空链表。

    \item  我再创建了一个list2,通过\texttt{List<int> list2 = {1, 2, 3, 4, 5};}这种赋值构造的方法,理应得到size为5的一个链表。

    \item 我再用\texttt{List<int> list3(list2)}这种方法测试拷贝构造函数,应该得到一个和list2一样的深度拷贝。

    \item 接着再看移动构造函数是否正确,首先用\texttt{std::move}把刚才的list2转化为右值,形参列表中接受右值引用的移动构造函数被调用。
    移动后list2被接管,理应会被赋为空链表。

    \item 接着测试接连赋值$a=b=c$,以及自我赋值$a=a$两种类型会不会导致错误。

    \item \textbf{在非空链表中测试迭代器},在此过程中,顺带测试\textbf{插入}。首先对空链表list1 push进一些元素,并打印出来,在打印的过程中也调用了begin与end,即让iterator进行了前置以及后置++以及--运算的测试(共四种),其中,由于--it的特殊性,在代码形式上稍有不对称的情况。

    \item 测试在指定位置进行insert,\texttt{list1.insert(++list1.begin(), 25)},意在第二个位置insert一个内容为25的结点。

    \item 测试pop以及erase(对象\textbf{暂且都是非空的列表})。看erase是否可在指定位置erase。

    \item 测试clear清空列表,并且尝试clear以后再进行一次,验证对于空链表clear的效果。

    \item 测试int之外的类别,比如string,简单验证仍然成立。

\end{enumerate}

\section{测试的结果}

使用List.cpp中的代码测试(\textbf{最后一部分暂且注释掉}),结果一切正常。输出如下:

\begin{verbatim}
    List 1 size: 0
    List 2 size: 5
    List 3 size: 5
    List 4 size: 5, List 2 size after move: 0
    List 2's size becomes: 5 again
    Now List2's elements are: 1 2 3 4 5
    After list3=list3, list3's elements are: 1 2 3 4 5
    Let's test push and pop
    List1 is (by ++i): 30 20 10 -5
    List1 is (by i++): 30 20 10 -5
    inverse List1 is (by --i): -5 10 20 30
    inverse List1 is (by i--): -5 10 20 30
    List1 after insert: 30 25 20 10 -5
    List1 after pop 1st element, then erase 2nd element and pop the last element: 25 10
    List1 after erase the last element: 25
    List1 size after clear: 0
    List1's content now :
    StringList 1 size: 0
    StringList 2 size: 4
    StringList 2 content: Hello this is Maythics
    StringList 2 content: Hello this is Maythics ' homepage
\end{verbatim}

我用\texttt{ valgrind --leak-check=full  ./List} 进行测试,发现没有发生内存泄露。

\section{bug报告}

\subsection{问题}

将List.cpp最后的部分解注释,再次测试,会出现段错误等问题。\

bug1(对迭代器未限制++与- -范围):\

\begin{enumerate}
    \item 构建一个空列表
    \item 调用end获取迭代器iterator
    \item 迭代器进行++操作,并且打印*iterator
\end{enumerate}

会出现段错误:
\begin{verbatim}
    After illegal move:
    Segmentation fault
\end{verbatim}

原因是没有限制iterator的活动范围,应该在++与- -的函数处添加判断,使得iterator到达不了哨兵节点以外的地方(但是哨兵节点要允许到达,因为如循环打印出整个list的功能就需要iterator能够到达哨兵处比较)\


bug2(erase一个空链表),触发如下:

\begin{enumerate}
    \item 构建一个空列表
    \item 调用begin获取迭代器
    \item 调用erase删除该迭代器所指位置
\end{enumerate}

会出现段错误:
\begin{verbatim}
   Try to erase an empty list:
   Segmentation fault
\end{verbatim}

它出现的原因是:erase函数本身并未判断是否链表为空,如果真的为空,理应不做任何操作直接return,但是此时却尝试删除而出错。\

\subsection{解决方案}

将 \texttt{\#include "List.h" } 注释掉,并且改成\texttt{\#include "myList.h"},即可解决问题。

\end{document}

